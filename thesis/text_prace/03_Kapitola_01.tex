\chapter{Smithův normální tvar}
\setcounter{page}{1}
\pagenumbering{arabic}

V této kapitole se budeme zbývat definicí Smithova normálního tvaru (budeme značit \snf) celočíselných matic $\Zmat$, dokážeme jeho existenci pro libovolnou $A \in \Zmat$ a konečně uvedeme souvislost mezi \snf  a konečně generovanými komutativními grupami.

\begin{defi}Řekneme že matice A $ \in \Zmat $ je ve Smithově normálním tvaru jestliže
\begin{center}
$ A =
    \begin{pmatrix}
        q_1     & 0      & \cdots & \cdots & \cdots & \cdots & 0      \\
        0       & q_2    & \ddots &        &        &        & \vdots \\
        \vdots  & \ddots & \ddots & \ddots &        &        & \vdots \\
        \vdots  &        & \ddots & q_k    & \ddots &        & \vdots \\
        \vdots  &        &        & \ddots & 0      & \ddots & \vdots \\
        \vdots  &        &        &        & \ddots & \ddots & 0      \\
        0       & \cdots & \cdots & \cdots & \cdots & 0      & 0      \\
    \end{pmatrix}
$
\end{center}
a platí $q_i | q_{i+1}$ kde $i \in \{1, \dots, k-1\}$. Čísla $q_i$ pak nazýváme
\textit{invariantními faktory}.
\end{defi}


\begin{vet}[O Smithově normálním tvaru]
    Pro libovolnou celočíselnou matici $ B \in \Zmat $ existují invertibilní
    matice $ P,Q \in \Zmat $ a matice $ A $ ve Smithově normálním tvaru takové,
    že platí
    \begin{center}
        $ B = P \cdot A \cdot Q $
    \end{center}
    Smithův normální tvar je jednoznačný až na znaménka invariantních faktorů.
\end{vet}


Než se pustíme do samotného důkazu této věty, je dobré si uvědomit, jak vlastně
vypadají invertibilní celočíselné matice. To popisuje následující lemma.


\begin{lem}
    Buď $A \in \Zmat$. Pak je A invertiblní, právě tehdy když je čtvercová a
    $\textnormal{det}(A) = \pm1$.
\end{lem}
\begin{proof}
    Buď $A \in \Zmat$ invertibilní. Existuje tedy matice $A^{-1} \in \Zmat$
    taková, že $AA^{-1} = E$. Pak je ovšem $A^{-1}$ inverzí pro $A$ také nad
    $\mathbb{Q}$. Proto $A$ musí být čtvercová, neboť každá invertiblní matice
    nad $\mathbb{Q}$ je čtvercová a má nenulový determinant. Navíc platí
    \begin{center}
        $det(A) \cdot \det(A^{-1}) = det(AA^{-1}) = det(E) = 1$
    \end{center}
    a protože determinant celočíselné matice je z definice determinantu
    také celočíselný, musí platit $det(A) = det(A^{-1}) = \pm1$ neboť v okruhu
    $\mathbb{Z}$ máme pouze dvě jednotky a to právě $\pm1$.

    Buď naopak  $A \in \Zmat$ čtvercová s determinantem $\pm1$. Pak inverzní
    matici $A^{-1}$ můžeme spočítat z algebraických doplňků jako
    \begin{center}
        $A^{-1} = \frac{1}{det(A)} \cdot A_{adj} = \pm A_{adj}$
    \end{center}
    nicméně prvky matice $A_{adj}$ - algebraické doplňky - se vypočítají
    ze subdeterminantů (minorů) matice A a musí být proto celočíselné. Matice
    $A^{-1}$ je tedy celočíselná.
\end{proof}

Z tohoto lemmatu tedy plyne, že pokud chceme celočíselnou matici B převést do
\snf pomocí invertibilních matic, musíme tak činit pouze prostřednictvím
matic majících determinant $\pm1$. Nyní tedy můžeme přikročit k důkazu samotné
věty o \snf.

\begin{proof}[Důkaz. (Věty o Smithově normálním tvaru)]
Nejprve dokážeme existenci \snf. Pro tento účel budeme potřebovat Euklidův
algoritmus. Ten funguje následujícím způsobem.

Pro libovolná $ a, b \in \mathbb{Z}$ taková, že $ |a| > |b| $ vydělíme číslo
$ a $ číslem $ b $ se zbytkem. Tedy $ a = qb + c $. Pak ovšem platí, že
$ gcd(a, b) = gcd(b, c) $ neboť

\begin{center}
$ gcd(a, b) = d \Rightarrow d \vert (a - qb) \Rightarrow d \vert c \Rightarrow d \vert gcd(b, c)$
\end{center}
a naopak
\begin{center}
$ gcd(b, c) = e \Rightarrow e \vert (qb + c) \Rightarrow e \vert a \Rightarrow e \vert gcd(a, b)$.
\end{center}
Takto můžeme postupovat rekurzivně a po konečném počtu kroků bude $ c = 0 $ a
$ b $ příslušné danému kroku bude právě hledaný největší společný dělitel.
Poznamenejme, že užití Euklidova algoritmu je z výpočetního hlediska výhodně, neboť
má logaritmickou složitost.

Dále protože výsledné transformační matice $ P, Q $ musí být invertibilní nad $\mathbb{Z}$,
plyne z předchozího lemmatu, že jejich determinant musí být roven $ \pm 1 $.
Evidentně tedy nemůžeme násobit řádek či sloupec matice jiným číslem než  $ \pm 1 $.
Můžeme však prohodit libovolné dva řádky, protože to lze realizovat pomocí
transformační matice,

\begin{center}
$
    \begin{pmatrix}
        1       &        &        &        &        &        &        \\
                & \ddots &        &        &        &        &        \\
                &        & 0      &        & 1      &        &        \\
                &        &        & \ddots &        &        &        \\
                &        & 1      &        & 0      &        &        \\
                &        &        &        &        & \ddots &        \\
                &        &        &        &        &        & 0      \\
    \end{pmatrix}
$
\end{center}
která má evidentně determinant roven $ -1 $. Analogicky můžeme prohazovat
prohazovat libovolné dva sloupce. A konečně pomocí transformační matice
\begin{center}
$
    \begin{pmatrix}
        1       &        &        &        &        &        &        \\
                & \ddots &        &        &        &        &        \\
                &        & 1      &        &        &        &        \\
                &        &        & \ddots &        &        &        \\
                &        & m      &        & 1      &        &        \\
                &        &        &        &        & \ddots &        \\
                &        &        &        &        &        & 1      \\
    \end{pmatrix}
$
\end{center}
můžeme k libovolnému řádku přičíst $ m $-násobek jiného řádku.

Nyní budeme postupovat následujícím způsobem. Na pozici $ (1, 1) $ přesuneme libovolný
nenulový prvek matice $ B $ (Pokud $ B = 0 $, pak je již ve \snf a žádné operace
provádět nemusíme). Pak postupně pro každý prvek pod a napravo od prvku $ b_1^1 $
aplikujeme Euklidův algoritmus (konkrétně jeho implementaci pomocí řádkových a
sloupcových operací, která potřebuje pouze operace násobení řádku/sloupce číslem $ -1 $,
přičítání násobku řádku/sloupce k jinému a prohazování dvou řádků/sloupců),
čímž na pozici $ (1, 1) $ vyrobíme největší společný prvků
v prvním sloupci a řádku. Tyto prvky můžeme tedy snadno vyeliminovat, čímž získáme
matici ve tvaru

\begin{center}
$ B =
    \begin{pmatrix}
        b_1^1   & 0      & \hdots & 0       \\
        0       & \ast   & \hdots & \ast    \\
        \vdots  & \vdots &        & \vdots   \\
        0       & \ast   & \hdots & \ast    \\
    \end{pmatrix}
$.
\end{center}


Pokud nyní existuje nějaký prvek $ b_j^i $, který ještě není dělitelný $ b_1^1 $, můžeme
přičíst $j$-tý sloupec k prvnímu sloupci a opět vyrobit na pozici $ (1, 1) $ prvek $ {b'}_{1}^{1} $
takový, že $ {b'}_{1}^{1} \vert b_1^1 $ a zároveň $ {b'}_{1}^{1} \vert b_j^i $,
který jej již dělit bude. Poznamenejme, že tento prvek bude nutně menší než původní
$ b_1^1 $, díky čemuž náš algoritmus skončí po konečném počtu kroků.

Celkem máme algoritmus, který převede matici $B$ do výše uvedeného tvaru
a navíc $ {b}_{1}^{1} \vert b_j^i $.
Označme takto vzniklou matici $ C $ a nechť $ q_k = c_1^1 $.
Nyní můžeme postupovat indukcí a aplikovat tento algoritmus
na submatici, která vznikne vynecháním prvního sloupce a řádku matice $C$.
Neboť $ q_k $ dělil všechny prvky matice $C$, bude dělit i prvek v levém
horním rohu submatice (označme jej $ q_{k+1} $) po aplikaci výše uvedeného algoritmu.
Dostáváme, že $ q_k \vert q_{k+1}$, což jsme měli dokázat.

Zbývá dokázat jednoznačnost. Označme
\begin{center}
$ gcd_{i \times i }(A) = gcd\{det(X) \vert X \textrm{ je submatice } A \textrm{ tvaru }i \times i\} $
\end{center}

Prvně ukážeme, že platí rovnost
\begin{center}
$ q_1 \hdots q_i = gcd_{i \times i }(A)$
\end{center}

kde $ A $ je matice ve \snf. Pokud submatice $ X $ obsahuje $k$-tý řádek, ale
neobsahuje $k$-tý sloupec matice A, bude její determinant evidentně nulový,
neboť $ A $ je diagonální a $ X $ tak bude obsahovat nulový řádek. Stačí tedy
uvažovat submatice jejichž diagonála leží na hlavní diagonále matice $ A $. To
znamená, že platí
\begin{center}
$ gcd_{i \times i }(A) = gcd\{q_{k_1} \hdots q_{k_i} \vert 1 \leq k_1 < \hdots < k_i \leq r \} $.
\end{center}
Navíc A je ve \snf, proto $ q_i | q_{i+1} $ z čehož plyne
\begin{center}
$ gcd\{q_{k_1} \hdots q_{k_i} \vert 1 \leq k_1 < \hdots < k_i \leq r \} = q_1 \hdots q_i $,
\end{center}
což jsme chtěli dokázat.

Konečně ukážeme, že největší společný dělitel subdeterminantů je invariantní
vzhledem k elementárním řádkovým operacím (invariance vzhledem k sloupcovým
operacím pak plyne ze symetrie).

Invariance vzhledem k násobení řádku číslem $ -1 $ a vzhledem k prohození řádků
je zřejmá, neboť tyto operace maximálně změní znaménko některých subdeterminantů.
To ovšem nemá žádný vliv na výsledného největšího společného dělitele. Pro přičítání
násobku řádku je situace ovšem poněkud složitější. Každý nový subdeterminant je
pak celočíselnou kombinací subdeterminantů předchozí matice. Z toho plyne, že
\begin{center}
$ gcd_{i \times i }(A) \vert gcd_{i \times i }(A') $.
\end{center}

Jak jsme ale ukázali dříve, operace přičtení řádku je invertibilní. Můžeme tedy
celý proces zopakovat opačným směrem a stejnou argumentací dostáváme
\begin{center}
$ gcd_{i \times i }(A') \vert gcd_{i \times i }(A) $.
\end{center}
Největší společný dělitel subdeterminantů se tedy nezmění.

Předpokládejme nyní, že \snf není jednoznačný a existují matice $ A,C $ a $ P,Q,T,U$
takové, že platí $ B = P \cdot A \cdot Q = T \cdot C \cdot U $, kde $ A,C $ jsou
různé a ve \snf a $ P,Q,T,U$ jsou celočíselné invertibilní matice. Pak násobení
invertibilními maticemi $ P,Q,T,U$ odpovídá postupnému provádění elemntárních
řádkových a sloupcových úprav, o kterých jsme ovšem dokázali, že nemění největšího
společného dělitele subdeterminantů. To speciálně znamená, že hlavní minory matic
$ A,C $ ve Smithově normálním tvaru jsou si rovny a proto i invariatní faktory
musí být stejně. To je spor s předpokladem. Smithův normální tvar je tedy jednoznačný.

\end{proof}
\section{Podkapitola}

\lipsum[98-105]

\subsection{Odstavec}

\lipsum[140-145]
\shorthandon{-}
%%%%%%%%%%%%%%%%%%%%%%%%%%%%%%%%%%%%
