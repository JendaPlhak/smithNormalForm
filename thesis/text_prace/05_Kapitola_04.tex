\chapter{Paralelizace}

Buď $ A $ celočíselná matice $ n \times m $ a předpokládejme, že $ n < m $. Pak
z Hadamardovy nerovnosti (pro bližší infromace čtenáře odkazujeme na publikaci
\cite{Hadamard}) plyne, že $ \det(A) \leq (m^{1/2} \Vert A \Vert )^m $, kde
$ \Vert A \Vert $ značí nejmenší celé číslo takové, že
$ \vert a^i_j \vert \leq \Vert A \Vert $. Z článku \cite{Had_tight} plyne, že
Hadamardův odhad je poněkud pesimistický a v případě náhodných matic vychází
determinant v průměru poněkud lépe, přesto však tento odhad dává tušit, že
se v případě počítačové implementace algoritmu pro výpočet Smithova normálního
tvaru můžeme poměrně rychle dostat do problémů s kapacitou celočíselných typů.

To samozřejmě můžeme řešit specializovanými knihovnami, které
reprezentují celé číslo jako třídu zaobalující pole celých čísel. Nicméně
efektivita takového řešení už není ideální a navíc tento přístup zvyšuje
paměťovou náročnost. Hodilo by se nám proto, kdybychom mohli celý výpočet
rozložit na více částí, pro které už by nebylo problém dosáhnout výsledku pomocí
standardní integerovské aritmetiky a následně všechny částečné vysledky spojit
do hledaného Smithova normálního tvaru. Toho můžeme dosáhnout pomocí Čínské
zbytkové věty.


\begin{vet}[Čínská zbytková věta] \label{Chin_Rem}
Mějme kladná celá čísla $ m_1,\dots,m_k $, která jsou po dvou nesoudělná. Pak
pro libovolnou posloupnost celých čísel $ a_1,\dots,a_k $ existuje nějaké celé
číslo $ x $, které je řešením následující soustavy kongruencí.

\begin{equation}
    \begin{aligned} \label{congrs}
        x &\equiv a_1 \pmod{m_1}                   \\
        x &\equiv a_2 \pmod{m_2}                   \\
          &\mathrel{\makebox[\widthof{=}]{\vdots}}  \\
        x &\equiv a_k \pmod{m_k}                   \\
    \end{aligned}
\end{equation}
%
Navíc pro libovolná dvě řešení $ x_1, x_2 $ uvedené soustavy platí
$ x_1 \equiv x_2 \pmod{m_1 m_2 \cdots m_k} $.
\end{vet}

\begin{proof}
Zaměřme svou pozornost nejprve na existenci řešení. Označme
$ M = m_1 m_2 \cdots m_k $. Pak zřejmě pro každé $ i \in \{1,\dots,k\} $
platí $ \gcd(m_i, M / m_i) = 1 $, protože $ m_i $ jsou po dvou nesoudělná.
Díky tomu můžeme pomocí rozšířeného Euklidova algoritmu najít celá čísla
$ s_i, t_i $ taková, že $ s_i m_i + t_i \frac{M}{m_i} = 1 $. Nyní označme
$ d_i = t_i \frac{M}{m_i} $. Z toho plyne rovnost $ s_i m_i + d_i = 1 $ a navíc
\begin{align*}
d_i \equiv
\left\{
  \begin{array}{lr}
    1 \pmod{ m_j } & : j = i \\
    0 \pmod{ m_j }  & : j \neq i
  \end{array}
\right.
\end{align*}
Skutečně, pokud $ j = i $, tak dostáváme
$ 1 \equiv s_i m_i + d_i \equiv d_i \pmod{m_i} $. Pokud naopak $ j \neq i $, pak
zřejmě $ m_j \mid d_i $ a proto bude $ d_i $ kongruentní s nulou modulo $ m_j $.

Položme $ x = \sum\limits_{i = 1}^{k} a_i d_i $. Pak díky výše uvedené
vlastnosti $ d_i $ bude $ x $ řešením soustavy \eqref{congrs}.

Nechť pro nějaká dvě řešení soustavy \eqref{congrs} platí
$ x_1 \not\equiv x_2 \pmod{m_1 m_2 \cdots m_k} $. Pak ale
$ x_1 \equiv x_2 \pmod{m_i} \Rightarrow m_i \mid x_1 - x_2 $. Protože $ m_i $
jsou po dvou nesoudělná, tak z předchozí implikace dále plyne, že
$ x_1 - x_2 $ bude dělitelné také součinem všech $ m_i $. Tedy
$ M \mid x_1 - x_2 $. To je ovšem spor s předpokladem, který je ekvivalentní
tomu, že $ M \nmid x_1 - x_2 $.
\end{proof}

\begin{dus}\label{Chi_Iso}
Mějme kladné číslo $ m $ s faktorizací $ m = p_1^{r_1} \cdots p_k^{r_k}$. Pak
existuje isomorfismus okruhů
$ \Zbb / m \Zbb \cong \Zbb / p_1^{r_1} \Zbb \times \cdots \times \Zbb / p_k^{r_k} \Zbb $
\end{dus}
\begin{proof}

\end{proof}