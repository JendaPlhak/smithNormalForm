\chapter{Výpočet Smithova normálního tvaru trojúhelníkových matic}

V této sekci navážeme na předchozí kapitolu a uvedeme co to je Hermitův normální
tvar a jaký je jeho vztah k redukovanému schodovitému tvaru.
Těchto výsledků pak využijeme pro vytvoření algoritmu, který vypočítá
Smithův normální tvar právě z Hermitova normálního tvaru.
Při popisu algoritmů a výsledků v této kapitole budeme vycházet zejména z článku
,,\textit{Computing Hermite and Smith normal forms of triangular integer matrices}''
\cite{SNF_Arne}, který v roce 1998 publikoval Arne Storjohann.


\section{Hermitův normální tvar}

\begin{defi}\label{hermi}
    Nechť $ H \in Mat_{n \times n}\mathbb{Z} $ je matice mající plnou hodnost,
    kterou lze zapsat následujícím způsobem:
    \begin{align}
        H =
        \begin{pmatrix}
            h_1 & h_{12} & \hdots & h_{1n} \\
                & h_2    & \hdots & h_{2n} \\
                &        & \ddots & \vdots \\
                &        &        & h_n    \\
        \end{pmatrix}.
    \end{align}
    Pak $ H $ je v Hermitově normálním tvaru, právě když splňuje následující
    podmínky:
    \begin{Cond}
        \item Pro každé $ j \in \{1,\dots, n\}$ platí $ h_j > 0 $ a zároveň
        \item $ 0 \leq h_{ij} < h_j $ pro všechna $ 1 \leq i < j \leq n $.
    \end{Cond}
\end{defi}
\begin{pozn}
    Matice splňující požadavky předchozí definice jsou tedy horní trojúhelníkové
    a regulární. Navíc pro ně platí, že prvky nad diagonálou jsou nezáporné a
    shora omezené prvkem na diagonále, který musí být kladný.
\end{pozn}

V článku \cite{SNF_Arne} autor dále rozvádí, jakým způsobem je možné vypočítat
Hermitův normální tvar z horní trojúhelníkové matice, jejíž prvky jsou omezeny
determinantem. Tím se však v tomto textu nemusíme zabývat, neboť lze snadno
nahlédnout, že máme-li matici v redukovaném schodovitém tvaru a vhodným způsobem
přeskupíme její sloupce, dostaneme matici jejíž hlavní čtvercová submatice bude
splňovat podmínky Hermitova normálního tvaru. Toto schéma detailněji popíšeme
na konci této kapitoly a navíc uvedeme algoritmus pro výpočet \snf{} čtvercové
matice v Hermitově normálním tvaru.


\section{Sloupcová eliminace}

Buď $ T $ $ k \times m $ matice mající hodnost $ k $, jejíž prvních $ k - 1 $
sloupců je ve Smithově normálním tvaru. Matici $ T $ můžeme schématicky zapsat
následujícím způsobem:

\begin{align} \label{T_matrix}
T =
     \left(
    \begin{array}{ccccc|ccc}
        a_1 &     &        &         & t_1     & \ast   & \hdots & \ast   \\
            & a_2 &        &         & t_2     & \vdots &        & \vdots \\
            &     & \ddots &         & \vdots  & \vdots &        & \vdots \\
            &     &        & a_{k-1} & t_{k-1} & \vdots &        & \vdots \\
            &     &        &         & t_k     & \ast   & \hdots & \ast   \\
    \end{array}
        \right)
.
\end{align}

Cílem této sekce bude dokázat následující tvrzení.

\begin{vet}[Sloupcová eliminace] \label{Sloup_elim}
Nechť $ T $ je matice, kterou lze zapsat stejně jako v rovnosti \ref{T_matrix}
a navíc splňuje následující podmínky:
\begin{Cond}[series=Sloup_elim_CONDS]
    \item hlavní $ k \times k $ submatice má plnou hodnost,
    \item prvních $ k - 1 $ sloupců je ve Smithově normálním tvaru,
    \item prvky $ t_i $, $ i \in \{1,\dots, k-1\} $, jsou redukovány modulo $ t_k $.
\end{Cond}
Pak existuje deterministický algoritmus, který pomocí ekvivalentních řádkových
a sloupcových operací převede hlavní $ k \times k $ submatici $ T $ do
Smithova normálního tvaru.
\end{vet}

Pro lepší čitelnost rozdělíme důkaz této věty do několika lemmat, které na konci
této kapitoly spojíme v kompletní důkaz.

\begin{lem} \label{Sloup_elim_GCD}
Buď $ T $ matice splňující stejné podmínky jako v předpokladech věty
\ref{Sloup_elim} a navíc nechť $ k > 1 $. Pak existuje deterministický
algoritmus, který převede matici $ T $ na ekvivalentní matici splňující stejné
podmínky, ale navíc bude platit
\begin{Cond}[resume=Sloup_elim_CONDS]
    \item $ \gcd(a_i, t_i) = \gcd(a_i, t_i, t_{i+1},\dots, t_k) $ pro
        $ 1 \leq i \leq k - 1 $.
\end{Cond}
\end{lem}
\begin{proof}
Algoritmus bude pracovat iterativně vzhledem k proměnné $ r $, která bude značit
aktuálně zpracovávaný řádek. Je zřejmé, že pro $ r = k $, bude platit
\begin{align} \label{gcd_eq_induction}
    \gcd(T_{r,r}, T_{r,k}) = \gcd(T_{r,r}, T_{r,k},T_{r+1,k},\dots, T_{k,k}).
\end{align}
Můžeme tedy předpokládat, že pro nějaké $ i $, $ 1 \leq i < k $, splňuje
matice $ T $ rovnost \ref{gcd_eq_induction} pro všechna $ r = k, k-1,\dots, i + 1 $.
Nyní ukážeme, jakým způsobem aplikovat ekvivalentní řádkové a sloupcové operace
na matici $ T $ tak, aby výsledná matice splňovala podmínky (c1)-(c4) a platila
rovnost \ref{gcd_eq_induction} pro $ r = k, k-1,\dots, i $.

Buď $ 0 \leq c < a_i $ řešením rovnosti
$ \gcd(t_i + ct_{i+1}, a_i) = \gcd(t_i, t_{i+1}, a_i) $.
Takovéto $ c $ můžeme získat z důsledku \ref{Bach_Dus}. Přičtením $ c $-násobku
řádku $ i + 1 $ k $ i $-tému řádku $ \operatorname{row}(T, i) \pluseq \operatorname{row}(T, i + 1) $
získáme matici $ T' $ ve tvaru

\begin{align*}
T' =
    \left(
    \begin{array}{ccccccc|ccc}
        a_1 &        &     &          &        &         & t_1              & \ast   & \hdots & \ast   \\
            & \ddots &     &          &        &         & \vdots           & \vdots &        & \vdots \\
            &        & a_i & ca_{i+1} &        &         & t_{i} + ct_{i+1} & \vdots &        & \vdots \\
            &        &     & a_{i+1}  &        &         & t_{i+1}          & \vdots &        & \vdots \\
            &        &     &          & \ddots &         & \vdots           & \vdots &        & \vdots \\
            &        &     &          &        & a_{k-1} & t_{k-1}          & \vdots &        & \vdots \\
            &        &     &          &        &         & t_k              & \ast   & \hdots & \ast   \\
    \end{array}
    \right)
.
\end{align*}
Nyní počítejme:
\begin{align*}
    \gcd(T'_{i,i}, T'_{i,k}) &= \gcd(a_i, t_{i} + cT_{i+1})  \\
                             &= \gcd(a_i, t_{i}, t_{i+1}) \\
                             &= \gcd(a_i, t_{i}, a_{i+1}, t_{i+1}) \\
                             &= \gcd(a_i, t_{i}, t_{i+1},\dots,t_{k}),
\end{align*}
přičemž předposlední rovnost plyne z toho, že $ a_i \vert a_{i+1} $ a
poslední dostaneme z indukčního předpokladu. Rovnost \ref{gcd_eq_induction} je
tedy splněna.

Nyní stačí jen redukovat prvky v $ i $-tém řádku napravo od $ a_i $. Díky tomu,
že $ a_i \vert a_{i+1} $, bude $ i + 1 $ prvek $ ca_{i+1} $ vynulován a
matice sestavená z prvních $ k - 1 $ sloupců matice $ T' $ zůstane nezměněna.
\end{proof}

\begin{lem} \label{Sloup_elim_SNF}
Buď $ T $ matice splňující stejné podmínky jako v požadavcích lemmatu \ref{Sloup_elim_GCD}
a nechť je navíc splněna podmínka (c4). Pak existuje deterministický
algoritmus, který převede matici $ T $ na ekvivalentní matici splňující stejné
podmínky, ale navíc bude platit
\begin{Cond}[resume=Sloup_elim_CONDS]
    \item $ T_{1,1} $ dělí všechny prvky v prvních $ k $ sloupcích matice $ T $,
    \item $ T_{1,k} = 0 $.
\end{Cond}
\end{lem}
\begin{proof}
Našim cílem je transformovat matici $ T $ na ekvivalentní matici
\begin{align*}
T' =
    \left(
    \begin{array}{ccccc|ccc}
        s_1 &     &        &         &                 & \ast   & \hdots & \ast   \\
            & a_2 &        &         & t_2 a_1/s_1     & \vdots &        & \vdots \\
            &     & \ddots &         & \vdots          & \vdots &        & \vdots \\
            &     &        & a_{k-1} & t_{k-1} a_1/s_1 & \vdots &        & \vdots \\
            &     &        &         & t_k a_1/s_1     & \ast   & \hdots & \ast \\
    \end{array}
    \right),
\end{align*}
kde $ s_1 = \gcd(a_1, t_1) $. Nechť je trojice $ (s, t, s_1) $ řešením Bezoutovy
rovnosti $ s a_1 + t t_1 = s_1 $. Koeficienty $ (s, t, s_1) $ můžeme nalézt
například pomocí rozšířeného Euklidova algoritmu.
Pomocí těchto čísel definujme $ m \times m $ matici $ V $ jakožto
\begin{align*}
V =
    \left(
    \begin{array}{c|c|c|c}
      s      & \hdots  & - t_1 / s_1 &         \\ \hline
      \vdots & I_{k-2} & \vdots      &         \\ \hline
      t      & \hdots  & a_1 / s_1   &         \\ \hline
             &         &             & I_{m-k} \\
    \end{array}
    \right)
.
\end{align*}
Taková matice bude mít zřejmě determinant roven $ \pm 1 $ a bude tedy
unimodulární. Nyní tuto transformační matici aplikujeme na matici $ T $.
Výsledek této operace můžeme zapsat jako

\begin{align*}
TV =
    \left(
    \begin{array}{ccccc|ccc}
        s_1        &     &        &         &                 & \ast   & \hdots & \ast   \\
        tt_2       & a_2 &        &         & t_2 a_1/s_1     & \vdots &        & \vdots \\
        \vdots     &     & \ddots &         & \vdots          & \vdots &        & \vdots \\
        tt_{k - 1} &     &        & a_{k-1} & t_{k-1} a_1/s_1 & \vdots &        & \vdots \\
        tt_k       &     &        &         & t_k a_1/s_1     & \ast   & \hdots & \ast \\
    \end{array}
    \right)
.
\end{align*}
Prvek v prvním řádku a $ k $-tém sloupci bude skutečně nulový neboť
$ a_1(-\frac{t_1}{s_1}) + t_1 \frac{a_1}{s_1} = 0 $. Z podmínky (c4) navíc plyne,
že $ s_1 \vert t_i $ pro $ i = 1,\dots, k $ neboť $ s_1 = \gcd(a_1, t_1) $.
Díky tomu můžeme vyeliminovat všechny prvky pod diagonálou v prvním sloupci.
To že $ s_1 \vert t_i $ nám navíc zajišťuje splnění podmínky (c5), čímž je lemma
dokázáno.
\end{proof}

Nyní můžeme přistoupit k samotnému důkazu věty \ref{Sloup_elim}.
\begin{proof}[Důkaz věty o sloupcové eliminaci]
Buď $ T $ matice $ k \times m $, která splňuje podmínky uvedné v předpokladech
věty. Pokud $ k = 1 $, pak se naše submatice sestává z jediného prvku, je tedy
triviálně ve Smithově normálním tvaru a náš algoritmus může skončit. Nechť je
tedy $ k > 1 $. Na matici $ T $ nejdříve aplikujeme algoritmus lemmatu
\ref{Sloup_elim_GCD}. Tím zajistíme splnění předpokladů pro postupnou aplikaci
algoritmu lemmatu \ref{Sloup_elim_SNF}. Algoritmus budeme aplikovat na čtvercové
submatice $ i \times i $, pro $ i = k,k-1,\dots, 2 $, které leží na diagonále
původní matice $ T $ a pravý dolní roh se překývá s prvkem $ T_{k,k} $
(v angličitně je nazýváme trailing matrices).

Takto zajistíme úpravu matice $ T $ do vhodného tvaru a nakonec zbývá jen
vhodným způsobem pronásobit $ k $-tý řádek $ -1 $ tak, aby platilo
$ T_{k,k} = |T_{k,k}| $. Hlavní submatice o rozměrech $ k \times k $ pak bude
ve Smithově normálním tvaru, čímž je věta dokázána.
\end{proof}