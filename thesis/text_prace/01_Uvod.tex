\chapter*{Úvod}
\addcontentsline{toc}{chapter}{Úvod}

Cílem této práce je seznámit čtenáře s efektivním algoritmem pro výpočet Smithova normálního tvaru celočíselných matic.

Pro potřeby tohoto textu musíme nejprve zavést některé pojmy, které
nám umožní snazší výklad některých algoritmů.

\begin{defi}\label{hlavni_submatice}
    Buď $ A $ matice $ n \times m $. \textit{Hlavní} $ k \times l $ submaticí matice $ A $
    budeme rozumět submatici o rozměrech $ k \times l $, jejíž levý horní roh
    je shodný s levým horním rohem matice $ A $.
    \textit{Trailing} $ k \times l $ submaticí matice $ A $ budeme rozumět submatici o
    rozměrech $ k \times l $, jejíž pravý dolní roh je shodný s pravým dolním
    rohem matice $ A $.
\end{defi}

\cleardoublepage

