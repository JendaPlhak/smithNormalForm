\chapter*{Úvod}
\addcontentsline{toc}{chapter}{Úvod}

Smithův normální tvar celočíselných matic nalézá uplatnění například při řešení
soustav celočíselných rovnic nebo při počítání simpliciálních homologií \cite{Simplic_Homo}.
ílem této práce proto bude seznámit čtenáře s efektivním
algoritmem pro výpočet Smithova normálního tvaru celočíselných matic.

V první kapitole definujeme Smithův normální tvar a ukážeme, že pro každou
celočíselnou matici takovýto tvar existuje a je jednoznačný. Vedlejším produktem
tohoto důkazu bude také naivní algoritmus.
Tento naivní přístup však nebude poskytovat žádná omezení na velikosti prvků
matic vznikajících během výpočtu. Z toho důvodu uvedeme také druhý algoritmus, který
bude možné pomocí Čínské zbytkové věty paralelizovat a poskytneme záruky na
bitovou délku prvků všech matic vyskytujících se v algoritmu.
Popis tohoto algoritmu bude náplní kapitol 2, 3 a 4.

Ve druhé kapitole popíšeme, jakým způsobeme můžeme pro celočíselnou matici
vypočítat ekvivalentní matici v redukovaném schodovitém tvaru. Třetí kapitola
na tento výsledek naváže a poskytne popis algoritmu pro výpočet Smithova
normálního tvaru z matice v redukovaném schodovitém tvaru. Ve čtvrté kapitole
nastíníme možnosti paralelního výpočtu pomocí Čínské zbytkové věty a rozebereme
jakým způsobem můžeme omezit bitovou délku prvků matic v průběhu výpočtu.

V první kapitole budeme vycházet zejména z učebního textu \cite{vokr}. V 
kapitolách 2 respektive 3 pak čerpáme z článků \cite{triang} respektive \cite{SNF_Arne}.

Předpokládá se, že čtenář má základní znalosti z lineární algebry a
elementární teorie čísel.

\cleardoublepage

