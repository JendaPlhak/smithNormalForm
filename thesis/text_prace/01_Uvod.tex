\chapter*{Úvod}
\addcontentsline{toc}{chapter}{Úvod}

Smithův normální tvar nalézá uplatnění například při výpočtu simpliciálních
homologií \cite{Simplic_Homo}. Vzniká tak potřeba Smithův normální tvar efektivně
algoritmicky počítat. Cílem této práce proto bude seznámit čtenáře s efektivním
algoritmem pro výpočet Smithova normálního tvaru celočíselných matic.

V první kapitole definujeme Smithův normální tvar a ukážeme, že pro každou
celočíselnou matici takovýto tvar existuje a je jednoznačný. Vedlejším produktem
tohoto důkazu bude také naivní algoritmus pro výpočet Smithova normálního tvaru.
Tento naivní přístup však nebude poskytovat žádná záruky ohledně velikosti prvků
pracovních matic. Z toho důvodu uvedeme také druhý algoritmus, který
bude možné pomocí Čínské zbytkové věty paralelizovat a poskytneme záruky na
bitovou délku prvků pracovních matic.
Ve druhé kapitole tak nejpve popíšeme, jakým způsobeme můžeme pro celočíselnou matici
vypočítat ekvivalentní matici v redukovaném schodovitém tvaru. Třetí kapitola
na tento výsledek naváže a poskytne popis algoritmu pro výpočet Smithova
normálního tvaru z matic v redukovaném schodovitém tvaru. Ve čtvrté kapitole
nastíníme možnosti paralelního výpočtu pomocí Čínské zbytkové věty a rozebereme
jakým způsobem můžeme omezit bitovou délku prvků pracovních matic.
Nakonec v páté kapitole popíšeme experiemtnální implementaci algoritmu pro
výpočet Smithova normálního tvaru v jazyce C++.

V první kapitole budeme vycházet zejména ze skipt \cite{vokr}. V kapitolách
3 respektive 4 pak čerpáme z článků \cite{triang} respektive \cite{SNF_Arne}.

Předpokládá se, že čtenář má základní znalosti z lineární algebry a teorie čísel.

\cleardoublepage

