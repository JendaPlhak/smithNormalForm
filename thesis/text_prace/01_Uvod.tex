\chapter*{Úvod}
\addcontentsline{toc}{chapter}{Úvod}

Smithův normální tvar nalézá uplatnění například při výpočtu simpliciálních
homologií \ref{Simplic_Homo}. Vzniká tak potřeba Smithův normální tvar efektivně
algoritmicky počítat. Cílem této práce proto bude seznámit čtenáře s efektivním
algoritmem pro výpočet Smithova normálního tvaru celočíselných matic.

V první kapitole definujeme Smithův normální tvar a ukážeme, že pro každou
celočíselnou matici takovýto tvar existuje a je jednoznačný. Vedlejším produktem
tohoto důkazu bude také naivní algoritmus pro výpočet Smithova normálního tvaru.
Ve druhé kapitole popíšeme, jakým způsobeme můžeme pro celočíselnou matici
vypočítat ekvivalentní matici v redukovaném schodovitém tvaru. Třetí kapitola
na tento výsledek naváže a poskytně popis algoritmu pro výpočet Smithova
normálního tvaru z matic v redukovaném schodovitém tvaru. V kapitolách 4 a 5
pak nastíníme možnosti paralelního výpočtu pomocí Čínské zbytkové věty a popíšeme
experiemtnální implementaci algoritmu pro výpočet Smithova normálního tvaru v
jazyce C++.

V první kapitole budeme vycházet zejména ze skipt \ref{vokr}. V kapitolách
3 respektive 4 pak čerpáme z článků \ref{triang} respektive \ref{SNF_Arne}.

Předpokládá se, že čtenář má základní znalosti z lineární algebry a teorie čísel.




\cleardoublepage

