\chapter*{Přehled použitého značení}
\addcontentsline{toc}{chapter}{Přehled použitého značení}

Pro snažší orientaci v textu zde čtenáři předkládáme přehled základního značení, které se v celé práci vyskytuje.
\begin{flushleft}
\begin{longtable}[l]{ll} %% [l] tabulka je zarovnana vlevo; [c] zarovnani na stred; [r] zarovnani v pravo
  $\Rbb$                      & množina všech reálných čísel \\[1mm]
  $\Zbb$                      & množina všech celých čísel \\[1mm]
  $\mathbb{P}$                & množina všech prvočísel\\[1mm]
  $\Zmat$                     & množina všech celočíselných matic\\[1mm]
  $\Zmats$                    & množina všech celočíselných matic o rozměrech $ n \times m $\\[1mm]
  $A_{i, j}$			      & prvek v $ i $-tém řádku a $ j $-tém sloupci matice $ A $\\[1mm]
  $\Vert A \Vert$             & maximum absolutních hodnot prvků matice $ A $, $ \Vert A \Vert = \max{\{\vert A_{i, j} \vert \}}$\\[1mm]
  $\operatorname{row}(A, i) $ & $ i $-tý řádek matice $ A $\\[1mm]
  $\operatorname{col}(A, i) $ & $ i $-tý sloupec matice $ A $\\[1mm]

\end{longtable}
\end{flushleft}

\cleardoublepage
