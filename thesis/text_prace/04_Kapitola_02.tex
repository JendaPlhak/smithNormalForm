\chapter{Triangularizace celočíselných matic}

V této kapitole se budeme zabývat popisem algoritmu pro výpočet redukovaného
schodovitého tvaru celočíselných matic. Tento algoritmus představil Arne Storjohann
v článku nazvaném ,,\textit{A fast+practial+deterministic algorithm for
triangularizing integer matrices}'' \cite{triang}. Definujme nejdříve tvar matice,
jehož vytvoření bude našim cílem.

\begin{defi}\label{RREF}
    Řekneme že matice $ A \in \Zmat $ je v redukovaném schodovitém tvaru (\rst) jestliže
    splňuje následující podmínky:
    \begin{Cond}
        \item Buď $ r $ hodnost matice A. Pak prvních $ r $ řádků je nenulových
        a zbylých $ n - r $ řádků je nulových.
        \item Pro každé $ 1 \leq i \leq r $ buď $A[i, j_i]$ první nenulový prvek v
        $ i $-tém řádku. Pak $ j_1 < j_2 < \cdots < j_r $.
        \item Pro každé $ 1 \leq i \leq r $ platí $ A[i, j_i] > 0 $.
        \item Pro každé $ 1 \leq k < i \leq r $ platí $ A[i, j_i] > A[k, j_i] \geq 0 $.
    \end{Cond}
\end{defi}

\begin{pozn}
Poznamenejme, že první a druhá podmínka nám zaručují schodovitý tvar matice $ A $.
Tento však zjevně není jednoznačný. Proto je nutné přidat ještě podmínky
(\textit{c3}) a (\textit{c4}). (\textit{c3}) zajišťuje, že členy nad pivoty budou
kladné a (\textit{c4}) říká, že prvky nad pivoty budou pivoty omezeny. Tyto
podmínky pak určují tvar matice $ A $ jednoznačně vzhledem k elementárním
operacím.
\end{pozn}

\begin{pri}
Pro ilustraci uvádíme následující matici v \rst:
\begin{center}
$
    \begin{pmatrix}
        2 & 33 & 6  & 0 & 39  & 73 \\
        0 & 0  & 24 & 0 & 444 & 8  \\
        0 & 0  & 0  & 1 & 22  & 23 \\
        0 & 0  & 0  & 0 & 0   & 75 \\
        0 & 0  & 0  & 0 & 0   & 0  \\
    \end{pmatrix}
$
\end{center}
\end{pri}

Pro další práci budeme potřebovat malé lemma, které budeme využívat
například při přičítání násobku řádku k jinému řádku. Tento výsledek uvedl
v roce 1992 E. Bach \cite{Bach}. Nejprve však dokažme následující tvrzení:

\begin{lem} \label{factorization}
    Mějme kladná celá čísla $ g > 1 $ a $ N $ taková, že $ g \mid N $. Pak
    existuje algoritmus, který najde kladná celá čísla $ X, Y $ taková, že
    $ N = XY $, $ \gcd(X, Y) = 1 $, $ \gcd(g, Y) = 1 $ a navíc požadujeme, aby
    pro libovolné prvočíslo $ p \in \mathbb{P} $ platilo
    $ p \mid X \Rightarrow p \mid g $.
\end{lem}
\begin{pozn}
Pro lepší pochopení tohoto lemmatu poznamenejme, že jde vlastně jen o vhodné
rozdělení prvočísel, které tvoří $ N $, do proměnných $ X $, $ Y $ tak, abychom
jich do $ X $ přesunuli pouze nezbytně mnoho.
\end{pozn}
\begin{proof}
    Uvažme následující algoritmus\\
    %
    \begin{algorithm}[H] \label{factorization_Algo}
        \Begin{
           $ X \leftarrow g $\;
           $ Y \leftarrow \frac{N}{g} $\;
           \While{ $ \gcd(X, Y) \neq 1 $ } {
                $ h \leftarrow \gcd(X, Y) $\;
                $ X \leftarrow X h $\;
                $ Y \leftarrow \frac{Y}{h} $\;
           }
        }
    \end{algorithm}

    Je zřejmé, že uvedený cyklus bude vždy konečný, neboť v každé iteraci platí
    $ h > 1 $ a proměnná $ Y $ se tak s každou iterací zmenší. Navíc pro každou
    iteraci platí $ Y > 0 $.

    Jistě bude platit $ N = XY $, neboť po prvním kroku platí
    $ XY = g \frac{N}{g} = N $ a v každém dalším kroku pouze přesouváme
    prvočísla z $ Y $ do $ X $.

    Dále protože $ Y $ je nesoudělné s $ X = gk $, kde $ k $ je vhodný koeficient,
    bude $ Y $ nesoudělné také s $ g $.

    Nakonec ještě ukážeme platnost požadované implikace. Ta bude po prvním kroku
    algoritmu platit triviálně. V každé další iteraci pak budeme do $ X $
    přidávat pouze mocniny prvočísel, která se v $ X $ už nachází, což na platnost
    implikace nebude mít žádný vliv.
\end{proof}
\begin{pozn}
Hledání čísel $ X $ a $ Y $ skutečně musíme provádět iterativně. Například pro
volbu $ g = 2 $ a $ N = 24 $ bychom po první iteraci dostali $ X = 4 $ a $ Y = 6 $,
což jsou zřejmě čísla soudělná.
\end{pozn}

A nyní již k samotnému Bachovu lemmatu.
\begin{lem} [Bach] \label{bach}
    Mějme celá čísla $ a, b $ a $ N $, pro která platí $ N > 0 $ a zároveň \newline
    $ \gcd(a,b,N) = 1 $. Pak existuje deterministický algoritmus, který pro
    čísla $ a, b $ a $ N $ vypočte celé číslo $ 0 \leq c < N $ takové, že
    $ \gcd(a + cb,N) = 1 $.
\end{lem}
\begin{proof}
Hledaný algoritmus můžeme zapsat následujícím způsobem:\\
    %
    \begin{algorithm}[H] \label{Bach_Algo}
        \Begin{
            \uIf {$ \gcd(a,N) = 1 $} {
                $ c \leftarrow 0 $\;
            }
            \uElseIf {$ \gcd(a + b, N) = 1 $} {
                $ c \leftarrow 1 $\;
            }
            \Else {
                $ g \leftarrow \gcd(a, N) $\;
                $ (X, Y) \leftarrow $ čísla z předchozího lemmatu
                \ref{factorization} pro proměnné $ g $ a $ N $\;

                $ c \leftarrow $ číslo $ 0 < c < N $ takové, že
                    $ c \equiv 1 \pmod{X} $ a zároveň $ c \equiv 0 \pmod{Y} $\;
            }
        }
    \end{algorithm}
Nyní ukážeme, že takto definovaný algoritmus je skutečně korektní. V prvních
dvou případech, kdy za $ c $ volíme $ 0 $ a $ 1 $, jsou podmínky
kladené na koeficient $ c $ evidentně splněny. Zaměřme tedy naši pozornost na třetí
možnost. Z $ \gcd(a,N) \neq 1 $ plyne, že $ g $ ve svém rozkladu obsahuje
alespoň jedno prvočíslo dělící $ N $, je tedy korektní aplikovat lemma
\ref{factorization} na čísla $ N $ a $ g $. Výsledná čísla $ X, Y $ budou
nesoudělná a koeficient $ c $ můžeme najít pomocí Čínské zbytkové věty. Ukážeme,
že $ c $, získané z uvedené soustavy kongruencí pro $ X $ a $ Y $, bude skutečně
splňovat požadavky lemmatu.

Zřejmě platí, že $ \gcd(a+cb, N) = 1 \Leftrightarrow a + cb $ není dělitelné žádným
prvočíslem $ p \in \mathbb{P}$, $p \mid N $. Pro další postup budeme potřebovat
následující implikace.

$ p \nmid a \Rightarrow p \mid Y $. Obměnou implikace dostáváme
$ p \mid a \Leftarrow p \nmid Y \Leftrightarrow p \mid X $. Avšak
$ p \mid X \Rightarrow p \mid g \Rightarrow p \mid a $, proto implikace platí.
Nechť naopak $ p \mid Y $. Pak ale
$ p \mid Y \Rightarrow p \nmid g \Rightarrow p \nmid a $. Celkem jsme dostali
ekvivalenci $ p \nmid a \Leftrightarrow p \mid Y $. Z toho navíc negací plyne, že
$ p \mid a \Leftrightarrow p \mid X $.

Pro spor předpokládejme, že existuje nějaké $ p \in \mathbb{P}$,
$ p \mid N $ a $ p \mid a + cb $.
Díky výše uvedeným ekvivalencím můžeme rozlišit dva případy.

(\textit{i}) $ p \nmid a \Leftrightarrow p \mid Y $.
Z předpokladu pak plyne, že $ a + cb \equiv 0 \pmod{p} $. Avšak z toho, že
$ p \mid Y $ a z požadavků, které klademe na $ c $ dostaneme
$ a + cb \equiv a \equiv 0 (\bmod p) $. To ovšem implikuje  $ p \mid a $, což
je spor.

(\textit{ii}) $ p \mid a \Leftrightarrow p \mid X $.
Podobně jako v předchozím případě platí $ a + cb \equiv a + b \equiv 0 (\bmod p) $.
Z toho plyne, že $ p \mid a + b $. Avšak předpokládali jsme, že $ p \mid a $.
Proto nutně $ p \mid b $. Pak ale $ p \mid \gcd(a,b,N) $, spor.

\end{proof}
\begin{pozn}
První dvě větve uvedeného algoritmu ve skutečnosti nejsou pro dokázání lemmatu
potřebné. Uvádíme je pouze z optimalizačních důvodů.
\end{pozn}
\begin{dus}\label{Bach_Dus}
Uvedený algoritmus můžeme snadno rozšířit na případy, kdy hledáme $ 0 < c < N $
takové, že $ \gcd(a + cb,N) = d $, kde $ d = \gcd(a,b,N) $.
\end{dus}
\begin{proof}
Pomocí algoritmu z lemmatu \ref{bach} můžeme najít řešení úlohy
$ \gcd(\frac{a}{d} + c\frac{b}{d},\frac{N}{d}) = 1 $. Takto získané $ c $ pak
jistě splní naše požadavky, neboť pronásobením předchozí rovnosti číslem $ d $
dostáváme $ d = d \gcd(\frac{a}{d} + c\frac{b}{d},\frac{N}{d}) =
\gcd(d(\frac{a}{d} + c\frac{b}{d}),d\frac{N}{d}) = \gcd(a + cb,N) $.
\end{proof}

V následujících podkapitolách nejdříve popíšeme několik klíčových procedur, které
budou upravovat vstupní matici $ A $ pomocí unimodulárních (mající determinant
roven $\pm 1$, tedy invertibilních) matic.
Tyto procedury postupně propojíme a v poslední podkapitole pak obdržíme samotný
algoritmus pro výpočet \rst.





\section{GCD redukce}
Jak jsme viděli již v důkazu věty o Smithově normálním tvaru, častou operací,
kterou s maticí při převodu do \snf provádíme, je eliminace všech prvků
nacházejících se pod nějakým námi zvoleným pivotem. Takováto eliminace je poměrně
náročná, neboť pro každý prvek musíme vytvářet největší společný dělitel s pivotem.
Bylo by proto výhodné, kdybychom mohli nějakým způsobem upravit prvky ve sloupci tak,
že největší společný dělitel nějakých dvou prvků daného sloupce bude zároveň největším
společným dělitelem všech prvků daného sloupce. A přesně to je obsahem následující věty.

\begin{vet}[GCD redukce] \label{gcd_red}
    Nechť $ B $ je celočíselná matice $ (k+2) \times 2 $ , $ rank(B) = 2 $, kterou
    můžeme zapsat jako
\begin{center}
$ B =
    \begin{pmatrix}
        N & \bar{N} \\
        a_0 & \bar{a}_0  \\
        b_1 & \bar{b}_1  \\
        \vdots & \vdots  \\
        b_k & \bar{b}_k  \\
    \end{pmatrix}
$,
\end{center}
kde N je kladné.
Pak existuje deterministický algoritmus, který pro matici $ B $ vypočte unimodulární
matici
\begin{center}
$ C =
    \begin{pmatrix}
        1 &    &     &        &     \\
          & 1  & c_1 & \cdots & c_k \\
          &    & 1   &        &     \\
          &    &     & \ddots &     \\
          &    &     &        & 1   \\
    \end{pmatrix}
$
\end{center}
takovou, že bude platit
\begin{center}
$ CB =
    \begin{pmatrix}
        N & \bar{N} \\
        a_k & \bar{a}_k  \\
        b_1 & \bar{b}_1  \\
        \vdots & \vdots  \\
        b_k & \bar{b}_k  \\
    \end{pmatrix}
\qquad \text{kde} \qquad
\begin{matrix}
    a_k &= a_0 + c_1 b_1 + \dots + c_k b_k \\
    \bar{a}_k &= \bar{a}_0 + c_1 \bar{b}_1 + \dots + c_k \bar{b}_k \\
\end{matrix}
$
\end{center}
a navíc $ CB $ bude splňovat následující podmínky:
\begin{Cond}
    \item hlavní submatice
    $
        \begin{pmatrix}
            N & \bar{N} \\
            a_k & \bar{a}_k  \\
        \end{pmatrix}
    $ je regulární a
    \item $ gcd(N, a_k) = gcd(N, a_0, b_1, b_2, \dots, b_k) $.
\end{Cond}
\end{vet}

\begin{proof}
Bez újmy na obecnosti můžeme předpokládat, že $ k > 0 $. Pokud by k bylo nulové,
můžeme zřejmě za $ C $ zvolit identitu, které splní naše požadavky.
Dále můžeme předpokládat, že hlavní submatice je regulární a tedy platí
$ N \bar{a}_0 - \bar{N} a_0 \neq 0 $. Pokud by tomu tak nebylo, přičteme k druhému
řádku nějaký řádek $ 2 < s \leq k + 2 $, pro který platí
$ N \bar{b}_s - \bar{N} b_s \neq 0 $.
Takový řádek jistě bude existovat, neboť matice $ B $ má plnou hodnost. Výsledná
matice pak bude mít hlavní submatici regulární. Pro takto upravenou matici můžeme
spočítat hledané koeficienty $ c_i $ a konečně ke koeficientu $ c_s $ přičteme $ 1 $,
což bude přesně odpovídat onomu přičtení $ s $-tého řádku, které jsme provedli na
začátku.

Nyní ukážeme, jak iterativně vypočítat $ c_l $ pro $ l = 1,\dots,k $. Označme
mezivýsledky našeho výpočtu následujícím způsobem:
\begin{equation}\label{oznac}
    \begin{split}
        a_l &= a_0 + c_1 b_1 + \dots + c_l b_l \\
        \bar{a}_l &= \bar{a}_0 + c_1 \bar{b}_1 +\dots+c_l \bar{b}_l
    \end{split}
\end{equation}
Po provedení kroku $ l-1 $ a na začátku kroku $ l $ jsou vypočítány koeficienty
$ c_1, \dots, c_{l-1} $ a jsou splněny podmínky
\begin{center}
    \begin{CondNum}
        \item $ gcd(N, a_i) = gcd(N, a_0, b_1, b_2, \dots, b_i) $
        \item $ N \bar{a}_i - \bar{N} a_i \neq 0 $
    \end{CondNum}
\end{center}
pro $ i = l - 1 $. Poznamenejme, že pro $ i = 0 $ jsou podmínky (1)
a (2) splněny triviálně. Teď musíme provést
indukční krok - najít vhodné $ c_l $ takové, že budou splněny podmínky
(1) a (2) pro $ i = l $.

Nechť $ g = gcd(a_{l-1}, b_l) $. Pak můžeme dělením se zbytkem najít celá
čísla $ q_1, q_2 $ a
$ 0 \leq \tilde{a}_{l-1}, \tilde{b}_{l} < N $ taková, že platí

\begin{equation}\label{division}
    \begin{split}
        a_{l - 1} / g & = q_1 N +  \tilde{a}_{l-1} \\
        b_{l} / g & = q_2 N +  \tilde{b}_{l}
    \end{split}
\end{equation}
Čísla $ \tilde{a}_{l-1} $ a $ \tilde{b}_{l} $ budou nesoudělná. Důkaz tohoto
tvrzení se ve skutečnosti redukuje na důkaz implikace
$ gcd(a,b) = 1 \Rightarrow gcd(a \mathbin{\%} N, b \mathbin{\%} N) = 1 $, kde
$ a \mathbin{\%} N $ značí zbytek po dělení $ a $ číslem $ N $. Tato implikace
skutečně platí, protože Bezoutovu rovnost pro $ a $ a $ b $ můžeme vzít modulo
$ N $, čímž dostaneme $ (a \mathbin{\%} N) t + (b \mathbin{\%} N) u = 1 $. Zbytky
po dělení číslem $ N $ proto budou nesoudělné.

Pomocí algoritmu, který jsme použili v důkazu lemmatu \ref{bach}, můžeme najít
kladné číslo $ t $ takové, že bude platit
\begin{equation}\label{t_search}
    gcd(\tilde{a}_{l-1} + t \tilde{b}_{l}, N) = 1
\end{equation}
a volbou $ c_l \leftarrow t $ zajistíme splnění podmínky (1). Skutečně:
\begin{align*}
    gcd(a_l, N) &= gcd(a_{l-1} + t b_l, N)  \\
                &= gcd(g (q_1 N +  \tilde{a}_{l-1})
                        + tg(q_2 N + \tilde{b}_{l}), N) \\
                &= gcd(g (\tilde{a}_{l-1} + t \tilde{b}_{l})
                        + g (q_1 + t q_2) N, N) \\
                &= gcd(g (\tilde{a}_{l-1} + t \tilde{b}_{l}), N) \\
                &= gcd(g, N) \\
                &= gcd(a_{l-1}, b_l, N) \\
                &= gcd(N, a_0, b_1, b_2, \dots, b_l) \\
\end{align*}
přičemž poslední rovnost plyne z indukčního předpokladu.

Nakonec musíme zajistit splnění i druhé podmínky (2). Buď $ l $ index aktuálního
kroku a uvažujme rovnost

\begin{equation}\label{det}
    \begin{vmatrix}
        N & \bar{N} \\
        a_{l-1} + x b_l & \bar{a}_{l-1} + x \bar{b}_l  \\
    \end{vmatrix}
    = 0.
\end{equation}
Tu můžeme upravit do tvaru
\begin{equation}
    (N \bar{b}_l - \bar{N} b_l) x = -(N \bar{a}_{l-1} - \bar{N} a_{l-1}).
\end{equation}
Poznamenejme, že výraz napravo je z indukčního předpokladu nenulový. Nyní můžeme
rozlišit dva případy.

    (\textit{i}) Výraz $ N \bar{b}_l - \bar{N} b_l $ je roven nule. V tom
případě nemůže existovat žádné $ x $, které by mohlo podmínku (2) pokazit.

(\textit{ii}) $ N \bar{b}_l - \bar{N} b_l \neq 0 $.
To implikuje, že prvek x je určen jednoznačně a můžeme jej vyjádřit jako
\begin{equation}\label{frac}
    x = -\frac{N \bar{a}_{l-1} - \bar{N} a_{l-1}} {N \bar{b}_l - \bar{N} b_l}
      \neq 0
\end{equation}
Pokud nám tedy v kroku \ref{t_search} výjde $ c_l $ různé od $ x $, je vše v pořádku.
Pokud ovšem $ c_l = t = x $, nebyla by podmínka (2) splněna.
To ale můžeme snadno napravit. Předpokládejme tedy, že $ 0 < x = t $.
Nechť $ \bar{t} $ je nejmenší nezáporné číslo, pro které platí
$ gcd(\tilde{a}_{l-1} + \bar{t} (-\tilde{b}_{l}), N) = 1 $. Volbou $ c_l \leftarrow -\bar{t}$
zajistíme splnění podmínky (2), protože $ c_l = -\bar{t} \leq 0 < x $. Platnost
podmínky (1) pro takovouto volbu $ c_l $ se pak dokáže zcela analogicky, jako
jsme to již provedli výše pro $ c_l = t $.
\end{proof}





\section{Sloupcová redukce}

V této části si ukážeme, jak využít výsledků předcházející věty \ref{gcd_red} k
eliminaci prvků ve sloupečku. Mějme tedy $ n \times 2 $
vstupní matici $ B $, kteroužto můžeme zapsat následujícím způsobem:
\begin{align} \label{B_col}
B =
    \begin{pmatrix}
        \ast   & \ast   \\
        \vdots & \vdots \\
        \ast   & \ast   \\
        N & \bar{N} \\
        a_0 & \bar{a}_0  \\
        b_1 & \bar{b}_1  \\
        \vdots & \vdots  \\
        b_k & \bar{b}_k  \\
    \end{pmatrix}
,
\end{align}
kde $ k \geq 0 $, $ N > 0 $ a trailing $ (k+2) \times 2 $ submatice má plnou
hodnost.

Našim cílem bude nalézt $ n \times n $ unimodulární matice
\begin{align} \label{Q_C}
C =
    \begin{pmatrix}
        1 &        &   &      &        &     \\
          & \ddots &   &      &        &     \\
          &        & 1 & \ast & \cdots & \ast \\
          &        &   & 1    &        &     \\
          &        &   &      & \ddots &     \\
          &        &   &      &        & 1   \\
    \end{pmatrix}
\quad \text{a} \quad
Q =
    \begin{pmatrix}
        1 &        &   & \ast   & \ast   &   &        &   \\
          & \ddots &   & \vdots & \vdots &   &        &   \\
          &        & 1 & \ast   & \ast   &   &        &   \\
          &        &   & \ast   & \ast   &   &        &   \\
          &        &   & \ast   & \ast   &   &        &   \\
          &        &   & \ast   & \ast   & 1 &        &   \\
          &        &   & \vdots & \vdots &   & \ddots &   \\
          &        &   & \ast   & \ast   &   &        & 1 \\
    \end{pmatrix}
,
\end{align}
které budou reprezentovat příslušné invertibilní operace takové, že součin matic
$ QCB $ můžeme psát jako
\begin{align} \label{QCB}
QCB =
    \begin{pmatrix}
        \ast   & \ast   \\
        \vdots & \vdots \\
        \ast   & \ast   \\
        t_1    & \ast   \\
               & t_2    \\
               & \ast   \\
               & \vdots \\
               & \ast   \\
    \end{pmatrix}
\end{align}
a budou splněny podmínky následující věty.

\begin{vet}[Sloupcová redukce] \label{col_red}
Mějme matici $ B \in Mat_{n \times 2}\mathbb{Z} $, kterou můžeme zapsat jako v
\ref{B_col} s tím, že $ k \geq 0$, $ N > 0 $ a trailing $ (k+2) \times 2 $
submatice má plnou hodnost. Pak existuje algoritmus \textbf{ColumnReduction}($B, k$),
který na vstupu vezme $ B $ a $ k $, a jako výstup vrátí $ n \times n $
matice $ C $ a $ Q $, které lze vyjádřit jako v \ref{Q_C}. Navíc bude platit, že
součin $ QCB $ lze psát jako \ref{QCB} a bude splňovat následující podmínky:

\begin{Cond}
    \item $ t_1 > 0 $ a $ t_2 > 0 $,
    \item prvky nad $ t_1 $ v prvním sloupci jsou nezáporné a shora omezené číslem
    $ t_1 - 1 $,
    \item prvky nad a pod $ t_2 $ ve druhém sloupci jsou nezáporné a shora omezené
    číslem $ t_2 - 1 $.
\end{Cond}

\end{vet}
\begin{proof}
Nejprve aplikujeme algoritmus věty \ref{gcd_red} o GCD redukci na submatici
matice B tvořenou posledními $ k + 2 $ řádky. Tím získáme transformační
$ (k + 2) \times (k + 2) $ matici $ C' $, kterou když vhodně vložíme do jednotkové
matice $ n \times n $, získáme hledanou matici $ C $, která bude splňovat naše
požadavky. Konkrétně:
\begin{align*}
C =
    \begin{pmatrix}
        1 &        &   &    \\
          & \ddots &   &    \\
          &        & 1 &    \\
          &        &   & C' \\
    \end{pmatrix}
.
\end{align*}
Aplikací matice $ C $ na vstupní matici $ B $ dostáváme
\begin{align*}
CB =
    \begin{pmatrix}
        \ast   & \ast   \\
        \vdots & \vdots \\
        \ast   & \ast   \\
        N      & \bar{N} \\
        a_k    & \bar{a}_k  \\
        b_1    & \bar{b}_1  \\
        \vdots & \vdots  \\
        b_k    & \bar{b}_k  \\
    \end{pmatrix}
\end{align*}
s tím, že $ gcd(N, a_k) = gcd(N, a_k, b_1, b_2, \dots, b_k) $ a navíc submatice
\begin{align*}
    \begin{pmatrix}
        N      & \bar{N} \\
        a_k    & \bar{a}_k  \\
    \end{pmatrix}
\end{align*}
bude regulární.

Aplikací rozšířeného Euklidova algoritmu na dvojici $ (N, a_k) $ obdržíme
koeficienty $ t_1 $, $ m_1 $, $ m_2 $ takové, že
$ m_1 N + m_2 a_k = t_1 = gcd(N, a_k) $. Nyní můžeme vytvořit matici

\begin{align*}
U =
    \begin{pmatrix}
        m_1         & m_2      \\
        -sa_k / t_1 & sN / t_1 \\
    \end{pmatrix}
,
\end{align*}
kde $ s \in \{1, -1\} $ je zvoleno tak, aby
$ t_2 = (-sa_k / t_1) \bar{N} + (sN / t_1) \bar{a}_k $ bylo kladné. Matice $ U $
je unimodulární, neboť
\begin{align*}
\det U =
    \begin{vmatrix}
        m_1         & m_2      \\
        -sa_k / t_1 & sN / t_1 \\
    \end{vmatrix}
    = \frac{s(m_1 N + m_2 a_k)}{t_1} = \frac{s t_1}{t_1} = \pm 1
.
\end{align*}
A konečně můžeme zkonstruovat matici

\begin{align*}
Q =
    \begin{pmatrix}
        1 &        &   &   &   &        &   \\
          & \ddots &   &   &   &        &   \\
          &        & 1 &   &   &        &   \\
          &        &   & U &   &        &   \\
          &        &   &   & 1 &        &   \\
          &        &   &   &   & \ddots &   \\
          &        &   &   &   &        & 1 \\
    \end{pmatrix}
,
\end{align*}
která, jak plyne z předchozího, bude také unimodulární. Aplikací $ Q $ na matici
$ CB $ dostáváme
\begin{align*}
QCB =
    \begin{pmatrix}
        \ast   & \ast      \\
        \vdots & \vdots    \\
        \ast   & \ast      \\
        t_1    & \ast      \\
               & t_2       \\
        b_1    & \bar{b}_1 \\
        \vdots & \vdots    \\
        b_k    & \bar{b}_k \\
    \end{pmatrix}
\end{align*}
a platí $ t_1 \mid b_i$ kde $ i = 1, \dots, k $. Můžeme tedy snadno vyeliminovat
prvky pod $ t_1 $. Následně snadno provedeme redukci prvků nad $ t_1 $ a
konečně i prvků nad a pod $ t_2 $. To lze provést pomocí elementárních řádkových
operací, které můžeme odpovídajícím způsobem zapsat do matice $ Q $. Takto
upravená matice $ Q $ již bude splňovat podmínky věty a důkaz je hotov.

\end{proof}




\section{RST algoritmus}
V následujícím textu využijeme výše popsanou proceduru Sloupcové redukce k vytvoření
RST algoritmu, který na vstupu bere $ n' \times m' $ vstupní matici $ A' $ a
vrací její redukovaný schodovitý tvar včetně transformační matice. Nejdříve však
musíme definovat pojem rank profile (nejsme si vědomi existence vhodného českého
ekvivalentu, proto budeme používat původní anglický výraz).

\begin{defi}
Buď $ A $ matice $ n \times m $ a nechť $ r $ značí hodnost matice $ A $. Nechť
$ G $ je reprezentace matice $ A $ ve schodovitém tvaru. Pod pojmem
\textbf{rank profile} pak rozumíme uspořádanou $ r $-tici $ (j_1, \dots, j_r) $, kde
$ j_i $ je sloupcový index prvního nenulového prvku v $ i $-tém řádku matice $ G $.
\end{defi}

Abychom se vyhnuli ošetřování množství speciálních případů (například matice
mající hodnost $ 0 $ a podobně), budeme namísto matice $ A' $ uvažovat matici

\begin{align} \label{input_A}
A =
    \begin{pmatrix}
        1 &    &   \\
          & A' &   \\
          &    & 1 \\
    \end{pmatrix}
.
\end{align}
Poznamenejme, že takováto matice bude mít rank profile ve tvaru
$ (1, j_2, \dots, j_{r - 1}, m) $, kde $ r \geq 2 $ je hodnost matice $ A $ a
$ n \times m $ jsou její rozměry.

Nyní budeme definovat RST algoritmus. Pro názornost nejdříve uvedeme variantu,
která potřebuje dopředu znát rank profile. Ten je možné spočítat například
Gausovou eliminační metodou. Poznamenejme, že Gausova eliminace spadá do složitostní
třídy $ \mathcal{O}(n^3) $, což je zanedbatelné vzhledem k celkové složitosti
našeho algoritmu. Přesto však později uvedeme také jednoduchou modifikaci RST
algoritmu, která již rank profile nevyžaduje. Nyní přistupme k samotné definici.

\begin{algorithm} \label{RST_algorithm}
{
    \textbf{Algoritmus:} Výpočet redukovaného schodovitého tvaru
}

\KwData{
    Celočíselná $ n \times m $ matice $ A $ mající rank profile
    $ (j_1, \dots, j_r) $, kterou lze zapsat jako v \ref{input_A}.
}
\KwResult{
    Matice $ Q, C $ a $ T $ splňující $ QCA = T $, kde $ Q $ a $ C $ jsou
    unimodulární a $ T $ má prvních $ m - 1 $ sloupců v redukovaném schodovitém
    tvaru.
}
\Begin{
    $ Q^{(0)} \leftarrow I_n $\;
    $ C^{(0)} \leftarrow I_n $\;
    $ T^{(0)} \leftarrow A $\;
    % \vspace{\baselineskip}
    \For{ $k \leftarrow 1$ \KwTo $ r - 1 $} {
        $ B_k \leftarrow n \times 2 $ matice
            $ (\operatorname{col}(T^{(k - 1)}, j_k) \mid \operatorname{col}(T^{(k - 1)}, j_{k + 1})) $\;
        $ (\tilde{Q}, \tilde{C}) \leftarrow $ \textbf{ColumnReduction}$ (B_k, n - k - 1) $\;
        $ Q^{(k)} \leftarrow \tilde{Q} \tilde{C} Q^{(k - 1)} \tilde{C}^{-1} $\;
        $ C^{(k)} \leftarrow \tilde{C} C^{(k - 1)} $\;
        $ T^{(k)} \leftarrow \tilde{Q} \tilde{C} T^{(k - 1)} $\;
    }
    \Return{$ (Q^{(r - 1)}, C^{(r - 1)}, T^{(r - 1)}) $}
}

\end{algorithm}
\newpage

Abychom dokázali, že výše uvedený algoritmus je skutečně korektní, ukážeme nejdříve,
že matice $ Q^{(i)}, C^{(i)}, T^{(i)} $, které dostáváme v průběhu výpočtu, můžeme zapsat
následujícím způsobem:

\begin{align} \label{block_form}
    \stackrel{\mbox{$Q^{(i)}$}}{
        \left(
        \begin{array}{c|ccc|c}
          I_1 & &      & &  \\ \hline
              & &      & &  \\
              & & \ast & &  \\
              & &      & &  \\ \hline
              & &      & &  \\
              & & \ast & & I_{n - i - 1} \\
              & &      & &  \\
        \end{array}
        \right)
    }
    \stackrel{\mbox{$C^{(i)}$}}{
        \left(
        \begin{array}{c|ccc|c}
          I_1 & &      & &  \\ \hline
              & &      & &  \\
              & & \ast & & \ast \\
              & &      & &  \\ \hline
              & &      & &  \\
              & &      & & I_{n - i - 1} \\
              & &      & &  \\
        \end{array}
        \right)
    }
    A =
    \stackrel{\mbox{$T^{(i)}$}}{
        \left(
        \begin{array}{c|ccc|ccc}
          I_1 & &     &    & &      & \\ \hline
              & &     &    & &      & \\
              & & R_i &    & & \ast & \\
              & &     &    & &      & \\ \hline
              & &     &    & &      & \\
              & &     &    & & \ast & \\
              & &     &    & &      & \\
        \end{array}
        \right)
    }
\end{align}
pro $ i = 0, 1, \dots, r-1 $, kde $ R_i $ je matice v redukovaném schodovitém
tvaru.

\begin{lem} \label{RST_algo_lemma}
Pro všechna $ i = 0, 1, \dots, r - 1 $ můžeme matice $ Q^{(i)}, C^{(i)} $ a  $ T^{(i)} $
psát jako v \ref{block_form}, kde $ R_i $ je $ (i - 1) \times (j_{i + 1} - 2) $
matice v redukovaném schodovitém tvaru. Navíc bude platit:
\begin{Cond}
    \item $ Q^{(i)} $ a $ C^{(i)} $ jsou unimodulární a
    \item \label{RST_2} prvek $ T^{(i)}[i+1, j_{i+1}] $, který budeme značit $ N_{i + 1} $, bude
    nenulový a navíc bude platit $ 0 \leq T^{(i)}[l, j_{i+1}] < N_{i+1} $ pro
    $ l = 1, \dots, i, i+1, \dots, m $.
\end{Cond}
\end{lem}
\begin{proof}
Důkaz provedeme indukcí. V inicálním stavu algoritmu $ i = 0 $ jsou požadavky lemmatu
triviálně splněny, neboť $ Q^{(0)} = C^{(0)} = I_n $. Předpokládejme tedy, že
lemma platí pro $ i = k - 1 < r $, pro nějaké $ k $ kladné.
Dokážeme, že pak lemma platí také pro $ i = k $.

Submatici $ B_k $, tvořenou sloupci $ j_k $ a $ j_{k+1} $ matice
$ T^{(k-1)} $, můžeme psát jako
\begin{align*}
B_k =
    \begin{pmatrix}
        \ast   & \ast   \\
        \vdots & \vdots \\
        \ast   & \ast   \\
        N_k    & \bar{N}_k \\
        a_a    & \bar{a}_0  \\
        b_1    & \bar{b}_1  \\
        \vdots & \vdots  \\
        b_k'    & \bar{b}_k'  \\
    \end{pmatrix},
\end{align*}
kde $ k' = n - k - 1 $. Z indukčního předpokladu plyne, že submatice $ B_k $ má
následující dvě vlastnosti.
(1) $ N_k > 0 $, což plyne z podmínky \ref{RST_2} a toho, že $ N_k = T^{(k - 1)}[k, j_k] $.
A také (2) submatice složená z posledních $ k' + 2 $ řádků matice $ B_k $
bude mít plnou hodnost. To dostaneme z toho, že $ B_k $ je složená ze sloupců
$ j_k $ a $ j_{k+1} $ matice $ T^{(k - 1)} $ a navíc všechny prvky nalevo od
$ j_k $-tého sloupce v řádcích $k, k+1, \dots, m $ matice $ T^{(k - 1)} $ jsou nulové
(skutečně, jedná se o prvky, které se nachází pod blokem $ R_{k-1} $ z vyjádření
\ref{block_form}).

Vlastnosti (1) a (2) nám zaručují, že submatice $ B_k $ je validním vstupem pro
algoritmus \textbf{ColumnReduction}. Z něj pak získáme matice $ \tilde{Q} $ a
$ \tilde{C} $, které budou unimodulární. Navíc díky jejich specifické struktuře,
kterou nám garantuje věta \ref{col_red}, budou mít matice
$ { Q^{(k)} \leftarrow \tilde{Q} \tilde{C} Q^{(k - 1)} \tilde{C}^{-1} } $ a
$ { C^{(k)} \leftarrow \tilde{C} C^{(k - 1)} } $
strukturu zachycenou v rovnosti \ref{block_form}.

Nyní ukážeme, že rovnost \ref{block_form} je splněna pro $ i = k $
na konci $ k $-tého cyklu našeho algoritmu. To znamená ověřit rovnost
$ T^{(k)} = Q^{(k)} C^{(k)} A $. Poznamenejme, že matici $ T^{(k)} $ vypočítáme
jakožto $ T^{(k)} \leftarrow \tilde{Q} \tilde{C} T^{(k - 1)} $ a navíc
z indukčního předpokladu plyne, že platí $ T^{(k - 1)} = Q^{(k - 1)} C^{(k - 1)} A $.
To všechno nám dohromady dává následující rovnosti:

\begin{align*}
    T^{(k)} &= \tilde{Q} \tilde{C} T^{(k - 1)} \\
            &= \tilde{Q} \tilde{C} (Q^{(k - 1)} C^{(k - 1)} A) \\
            &= \tilde{Q} \tilde{C} (Q^{(k - 1)} (\tilde{C}^{-1} \tilde{C}) C^{(k - 1)} A) \\
            &= (\tilde{Q} \tilde{C} Q^{(k - 1)} \tilde{C}^{-1}) (\tilde{C} C^{(k - 1)}) A \\
            &= Q^{(k)} C^{(k)} A
\end{align*}
a rovnost \ref{block_form} je pro $ i = k $ skutečně splněna.

Nakonec uveďme, že díky struktuře součinu matic $ \tilde{Q} \tilde{C} B_k $, kterou
nám garantuje věta \ref{col_red}, a díky tomu, že submatice tvořená řádky
$ k, k+1, \dots, m $ a sloupci $ j_k, j_k + 1, \dots, j_{k+1} - 1 $ má hodnost
$ 1 $ (plyne z definice rank profile a indukčního předpokladu), bude možné matici
$ T^{(k)} = \tilde{Q} \tilde{C} T^{(k - 1)} $ schématicky zapsat jako v rovnosti
\ref{block_form} a navíc bude splňovat požadavky našeho lemmatu.
\end{proof}

RST algoritmus \ref{RST_algorithm} bere na vstupu libovolnou matici $ A' $ o
rozměrech $ n \times m $, která však musí byt vložena do $ (n + 2) \times (m + 2) $
matice $ A $ tak, jak je znázorňu je \ref{input_A}. Nyní ukážeme, že toto není
nikterak omezující, protože výstupní matice
($ Q^{(r - 1)} $, $ C^{(r - 1)} $, $ T^{(r - 1)} $) mohou být zachyceny
následujícím schématem:

\begin{align} \label{final_result}
    \stackrel{\mbox{$Q^{(r - 1)}$}}{
        \left(
        \begin{array}{c|ccc|c}
          I_1 & &      & &  \\ \hline
              & &      & &  \\
              & &  Q'  & &  \\
              & &      & &  \\ \hline
              & & \ast & & I_1 \\
        \end{array}
        \right)
    }
    \stackrel{\mbox{$C^{(r - 1)}$}}{
        \left(
        \begin{array}{c|ccc|c}
          I_1 & &      & &  \\ \hline
              & &      & &  \\
              & &  C'  & & \ast  \\
              & &      & &  \\ \hline
              & &      & & I_1 \\
        \end{array}
        \right)
    }
    \stackrel{\mbox{$A$}}{
        \left(
        \begin{array}{c|ccc|c}
          I_1 & &      & &  \\ \hline
              & &      & &  \\
              & &  A'  & &  \\
              & &      & &  \\ \hline
              & &      & & I_1 \\
        \end{array}
        \right)
    }
    =
    \stackrel{\mbox{$T^{(r - 1)}$}}{
        \left(
        \begin{array}{c|ccc|ccc}
          I_1 & &             & &      \\ \hline
              & &             & &      \\
              & &  R_{r - 1}  & & \ast \\
              & &             & &      \\ \hline
              & &             & & N_r  \\
        \end{array}
        \right)
    }
\end{align}

\begin{vet} \label{RST_algo}
Algoritmus pro výpočet redukovaného schodovitého tvaru je korektní. Výstupní
matice ($ Q^{(r - 1)} $, $ C^{(r - 1)} $, $ T^{(r - 1)} $) mají tvar,
který odpovídá rovnosti \ref{final_result}. Speciálně platí, že $ Q' $ a $ C' $
jsou unimodulární a splňují $ Q'C'A' = R_{r - 1} $, kde $ R_{r - 1} $ je
redukovaný stupňovitý tvar matice $ A' $.
\end{vet}
\begin{proof}
Platnost rovnosti \eqref{final_result} snadno plyne přímo z lemmatu
\ref{RST_algo_lemma} a rovnosti \ref{block_form} pro $ i = r - 1 $. Stejně tak
lze snadno odvodit, že $ Q' $ a $ C' $ jsou unimodulární a splňují
$ Q'C'A' = R_{r - 1} $.
\end{proof}


\cleardoublepage

