\chapter{Triangularizace celočíselných matic}

V této kapitole se budeme zabývat popisem algoritmu pro výpočet redukovaného
schodovitého tvaru celočíselných matic. Tento algoritmus představil Arne Storjohann
v článku nazvaném ,,\textit{A fast+practial+deterministic algorithm for
triangularizing integer matrices}'' \cite{triang}. Definujme nejdříve tvar matice,
jehož vytvoření bude našim cílem.

\begin{defi}\label{RREF}
    Řekneme že matice $ A \in \Zmat $ je v redukovaném schodovitém tvaru (\rst) jestliže
    splňuje následující podmínky:
    \begin{Cond}
        \item Buď r hodnost matice A. Pak prvních r řádků je nenulových.
        \item Pro každé $ 1 \leq i \leq r $ buď $A[i, j_i]$ první nenulový prvek v
        $ i $-tém řádku. Pak $ j_1 < j_2 < \cdots < j_r $.
        \item Pro každé $ 1 \leq i \leq r $ platí $ A[i, j_i] > 0 $.
        \item Pro každé $ 1 \leq k < i \leq r $ platí $ A[i, j_i] > A[k, j_i] \geq 0 $.
    \end{Cond}
\end{defi}

\begin{pozn}
Poznamenejme, že první a druhá podmínka nám zaručují schodovitý tvar matice $ A $.
Tento však zjevně není jednoznačný. Proto je nutné přidat ještě podmínky
(\textit{c3}) a (\textit{c4}). (\textit{c3}) zajišťuje, že členy nad pivoty budou
kladné a (\textit{c4}) říká, že prvky nad pivoty budou pivoty omezeny. Tyto
podmínky pak určují tvar matice $ A $ jednoznačně vzhledem k elementárním
operacím.
\end{pozn}

\begin{pri}
Pro ilustraci uvádíme následující matici v \rst:
\begin{center}
$
    \begin{pmatrix}
        2 & 33 & 6  & 0 & 39  & 73 \\
        0 & 0  & 24 & 0 & 444 & 8  \\
        0 & 0  & 0  & 1 & 22  & 23 \\
        0 & 0  & 0  & 0 & 0   & 75 \\
        0 & 0  & 0  & 0 & 0   & 0  \\
    \end{pmatrix}
$
\end{center}
\end{pri}

V následujících podkapitolách nejdříve popíšeme několik klíčových procedur, které
budou upravovat vstupní matici $ A $ pomocí unimodulárních (mající determinant
roven $\pm 1$, tedy invertibilních) matic.
Tyto procedury postupně propojíme a v poslední podkapitole pak obdržíme samotný
algoritmus pro výpočet \rst.





\section{GCD redukce}
Jak jsme viděli již v důkazu věty o Smithově normálním tvaru, častou operací,
kterou s maticí při převodu do \snf provádíme, je eliminace všech prvků
nacházejících se pod nějakým námi zvoleným pivotem. Takováto eliminace je poměrně
náročná, neboť pro každý prvek musíme vytvářet největší společný dělitel s pivotem.
Bylo by proto výhodné, kdybychom mohli nějakým způsobem upravit prvky ve sloupci tak,
že největší společný dělitel nějakých dvou prvků daného sloupce bude zároveň největším
společným dělitelem všech prvků daného sloupce. A přesně to je obsahem následující věty.

\begin{vet}[GCD redukce] \label{gcd_red}
    Nechť $ B \in \Zmat $ je matice $ (k+2) \times k $ a $ rank(B) = 2 $, kterou
    můžeme zapsat jako
\begin{center}
$ B =
    \begin{pmatrix}
        N & \bar{N} \\
        a_0 & \bar{a}_0  \\
        b_1 & \bar{b}_1  \\
        \vdots & \vdots  \\
        b_k & \bar{b}_k  \\
    \end{pmatrix}
$,
\end{center}
kde N je kladné.
Pak existuje deterministický algoritmus, který pro matici $ B $ vypočte unimodulární
matici
\begin{center}
$ C =
    \begin{pmatrix}
        1 &    &     &        &     \\
          & 1  & c_1 & \cdots & c_k \\
          &    & 1   &        &     \\
          &    &     & \ddots &     \\
          &    &     &        & 1   \\
    \end{pmatrix}
$
\end{center}
takovou, že bude platit
\begin{center}
$ CB =
    \begin{pmatrix}
        N & \bar{N} \\
        a_k & \bar{a}_k  \\
        b_1 & \bar{b}_1  \\
        \vdots & \vdots  \\
        b_k & \bar{b}_k  \\
    \end{pmatrix}
\qquad \text{kde} \qquad
\begin{matrix}
    a_k &= a_0 + c_1 b_1 + \dots + c_k b_k \\
    \bar{a}_k &= \bar{a}_0 + c_1 \bar{b}_1 + \dots + c_k \bar{b}_k \\
\end{matrix}
$
\end{center}
a navíc $ CB $ bude splňovat následující podmínky:
\begin{Cond}
    \item hlavní submatice
    $
        \begin{pmatrix}
            N & \bar{N} \\
            a_k & \bar{a}_k  \\
        \end{pmatrix}
    $ je regulární a
    \item $ gcd(N, a_k) = gcd(N, a_0, b_1, b_2, \dots, b_k) $.
\end{Cond}
\end{vet}

\begin{proof}
Bez újmy na obecnosti můžeme předpokládat, že $ k > 0 $. Pokud by k bylo nulové,
můžeme zřejmě za $ C $ zvolit identitu, které splní naše požadavky.
Dále můžeme předpokládat, že hlavní submatice je regulární a tedy platí
$ N \bar{a}_0 - \bar{N} a_0 \neq 0 $. Pokud by tomu tak nebylo, přičteme k druhému
řádku nějaký řádek $ 2 < s \leq k + 2 $, pro který platí
$ N \bar{b}_s - \bar{N} b_s \neq 0 $.
Takový řádek jistě bude existovat, neboť matice $ B $ má plnou hodnost. Výsledná
matice pak bude mít hlavní submatici regulární. Pro takto upravenou matici můžeme
spočítat hledané koeficienty $ c_i $ a konečně ke koeficientu $ c_s $ přičteme $ 1 $,
což bude přesně odpovídat onomu přičtení $ s $-tého řádku, které jsme provedli na
začátku.

Nyní ukážeme, jak iterativně vypočítat $ c_l $ pro $ l = 1,\dots,k $. Označme
mezivýsledky našeho výpočtu následujícím způsobem:
\begin{equation}\label{oznac}
    \begin{split}
        a_l &= a_0 + c_1 b_1 + \dots + c_l b_l \\
        \bar{a}_l &= \bar{a}_0 + c_1 \bar{b}_1 +\dots+c_l \bar{b}_l
    \end{split}
\end{equation}
Po provedení kroku $ l-1 $ a na začátku kroku $ l $ jsou vypočítány koeficienty
$ c_1, \dots, c_{l-1} $ a jsou splněny podmínky
\begin{center}
    \begin{CondNum}
        \item $ gcd(N, a_i) = gcd(N, a_0, b_1, b_2, \dots, b_i) $
        \item $ N \bar{a}_i - \bar{N} a_i \neq 0 $
    \end{CondNum}
\end{center}
pro $ i = l - 1 $. Poznamenejme, že pro $ i = 0 $ jsou podmínky (1)
a (2) splněny triviálně. Teď musíme provést
indukční krok - najít vhodné $ c_l $ takové, že budou splněny podmínky
(1) a (2) pro $ i = l $.

Nechť $ g = gcd(a_{l-1}, b_l) $. Pak můžeme dělením se zbytkem najít celá
čísla $ q_1, q_2 $ a
$ 0 \leq \tilde{a}_{l-1}, \tilde{b}_{l} < N $ taková, že platí

\begin{equation}\label{division}
    \begin{split}
        a_{l - 1} / g & = q_1 N +  \tilde{a}_{l-1} \\
        b_{l} / g & = q_2 N +  \tilde{b}_{l}
    \end{split}
\end{equation}
Čísla $ \tilde{a}_{l-1} $ a $ \tilde{b}_{l} $ jsou nesoudělná (snadno plyne z
Bezoutovy rovnosti). Pomocí
algoritmu uvedeného v TODO můžeme najít nejmenší kladné číslo $ t $ takové, že bude
platit
\begin{equation}\label{t_search}
    gcd(\tilde{a}_{l-1} + t \tilde{b}_{l}, N) = 1
\end{equation}
a volbou $ c_l \leftarrow t $ zajistíme splnění podmínky (1). Skutečně:
\begin{align*}
    gcd(a_l, N) &= gcd(a_{l-1} + t b_l, N)  \\
                &= gcd(g (q_1 N +  \tilde{a}_{l-1})
                        + tg(q_2 N + \tilde{b}_{l}), N) \\
                &= gcd(g (\tilde{a}_{l-1} + t \tilde{b}_{l})
                        + g (q_1 + t q_2) N, N) \\
                &= gcd(g (\tilde{a}_{l-1} + t \tilde{b}_{l}), N) \\
                &= gcd(g, N) \\
                &= gcd(a_{l-1}, b_l, N) \\
                &= gcd(N, a_0, b_1, b_2, \dots, b_l) \\
\end{align*}
přičemž poslední rovnost plyne z indukčního předpokladu.

Nakonec musíme zajistit splnění i druhé podmínky (2). Buď $ l $ index aktuálního
kroku a předpokládejme, že platí

\begin{equation}\label{det}
    \begin{vmatrix}
        N & \bar{N} \\
        a_{l-1} + x b_l & \bar{a}_{l-1} + x \bar{b}_l  \\
    \end{vmatrix}
    = 0
\end{equation}
pak ovšem z indukčního předpokladu plyne, že $ N \bar{b}_l - \bar{N} b_l \neq 0 $.
To implikuje, že prvek x je určen jednoznačně a můžeme jej vyjádřit jako
\begin{equation}\label{frac}
    x = -\frac{N \bar{a}_{l-1} - \bar{N} a_{l-1}} {N \bar{b}_l - \bar{N} b_l}
\end{equation}
Poznamenejme, že z indukčního předpokladu také plyne, že $ x \neq 0 $.

Pokud nám tedy v kroku \ref{t_search} výjde $ c_l $ různé od $ x $, je vše v pořádku.
Pokud ovšem $ c_l = t = x $, nebyla by podmínka (2) splněna.
To ale můžeme snadno napravit. Předpokládejme tedy, že $ 0 < x = t $.
Nechť $ \bar{t} $ je nejmenší nezáporné číslo, pro které platí
$ gcd(\tilde{a}_{l-1} + \bar{t} (-\tilde{b}_{l}), N) = 1 $. Volbou $ c_l \leftarrow -\bar{t}$
zajistíme splnění podmínky (2), protože $ c_l = -\bar{t} \leq 0 < x $. Platnost
podmínky (1) pro takovouto volbu $ c_l $ se pak dokáže zcela analogicky, jako
jsme to již provedli výše pro $ c_l = t $.
\end{proof}





\section{Sloupcová redukce}

V této části si ukážeme, jak využít výsledků předcházející věty \ref{gcd_red} k
eliminaci prvků ve sloupečku. Mějme tedy jako v předchozím $ n \times 2 $
vstupní matici $ B $, kteroužto můžeme zapsat následujícím způsobem:
\begin{align} \label{B_col}
B =
    \begin{pmatrix}
        N & \bar{N} \\
        a_0 & \bar{a}_0  \\
        b_1 & \bar{b}_1  \\
        \vdots & \vdots  \\
        b_k & \bar{b}_k  \\
    \end{pmatrix}
,
\end{align}
kde $ k \geq 0 $, $ N > 0 $ a trailing $ (k+2) \times 2 $ submatrix má plnou
hodnost (nejsme si vědomi českého ekvivalentu pro výraz trailing (sub)matrix,
a budeme jej proto v následujícím textu používát v nezměněné původní podobě).

Našim cílem bude nalézt $ n \times n $ unimodulární matice
\begin{align} \label{Q_C}
C =
    \begin{pmatrix}
        1 &        &   &      &        &     \\
          & \ddots &   &      &        &     \\
          &        & 1 & \ast & \cdots & \ast \\
          &        &   & 1    &        &     \\
          &        &   &      & \ddots &     \\
          &        &   &      &        & 1   \\
    \end{pmatrix}
\quad \text{a} \quad
Q =
    \begin{pmatrix}
        1 &        &   & \ast   & \ast   &   &        &   \\
          & \ddots &   & \vdots & \vdots &   &        &   \\
          &        & 1 & \ast   & \ast   &   &        &   \\
          &        &   & \ast   & \ast   &   &        &   \\
          &        &   & \ast   & \ast   &   &        &   \\
          &        &   & \ast   & \ast   & 1 &        &   \\
          &        &   & \vdots & \vdots &   & \ddots &   \\
          &        &   & \ast   & \ast   &   &        & 1 \\
    \end{pmatrix}
,
\end{align}
které budou reprezentovat příslušné invertibilní operace takové, že součin matic
$ QCB $ můžeme psát jako
\begin{align} \label{QCB}
QCB =
    \begin{pmatrix}
        \ast   & \ast   \\
        \vdots & \vdots \\
        \ast   & \ast   \\
        t_1    & \ast   \\
               & t_2    \\
               & \ast   \\
               & \vdots \\
               & \ast   \\
    \end{pmatrix}
\end{align}
a budou splněny podmínky následující věty.

\begin{vet}[Sloupcová redukce] \label{col_red}
Mějme matici $ B \in Mat_{n \times 2}\mathbb{Z} $, kterou můžeme zapsat jako v
\ref{B_col} s tím, že $ k \geq 0$, $ N > 0 $ a trailing $ (k+2) \times 2 $
submatrix má plnou hodnost. Pak existuje algoritmus \textbf{ColumnReduction}($B, k$),
který na vstupu vezme $ B $ a $ k $, a jako výstup vrátí $ n \times n $
matice $ C $ a $ Q $, které lze vyjádřit jako v \ref{Q_C}. Navíc bude platit, že
součin $ QCB $ lze psát jako \ref{QCB} a bude splňovat následující podmínky:

\begin{Cond}
    \item $ t_1 > 0 $ a $ t_2 > 0 $,
    \item prvky nad $ t_1 $ v prvním sloupci jsou nezáporné a shora omezené číslem
    $ t_1 - 1 $,
    \item prvky nad a pod $ t_2 $ ve druhém sloupci jsou nezáporné a shora omezené
    číslem $ t_2 - 1 $.
\end{Cond}

\end{vet}
\begin{proof}
Nejprve aplikujeme algoritmus věty \ref{gcd_red} o GCD redukci na submatici
matice B tvořenou posledními $ k + 2 $ řádky. Tím získáme transformační
$ (k + 2) \times (k + 2) $ matici $ C' $, kterou když vhodně vložíme do jednotkové
matice $ n \times n $, získáme hledanou matici $ C $, která bude splňovat naše
požadavky. Konkrétně:
\begin{align*}
C =
    \begin{pmatrix}
        1 &        &   &    \\
          & \ddots &   &    \\
          &        & 1 &    \\
          &        &   & C' \\
    \end{pmatrix}
.
\end{align*}
Aplikací matice $ C $ na vstupní matici $ B $ dostáváme
\begin{align*}
CB =
    \begin{pmatrix}
        \ast   & \ast   \\
        \vdots & \vdots \\
        \ast   & \ast   \\
        N      & \bar{N} \\
        a_k    & \bar{a}_k  \\
        b_1    & \bar{b}_1  \\
        \vdots & \vdots  \\
        b_k    & \bar{b}_k  \\
    \end{pmatrix}
\end{align*}
s tím, že $ gcd(N, a_k) = gcd(N, a_k, b_1, b_2, \dots, b_k) $ a navíc submatice
\begin{align*}
    \begin{pmatrix}
        N      & \bar{N} \\
        a_k    & \bar{a}_k  \\
    \end{pmatrix}
\end{align*}
bude regulární.

Aplikací rozšířeného Euklidova algoritmu na dvojici $ (N, a_k) $ obdržíme
uspořádanou trojici $ (t_1, m_1, m_2) $ takovou, že
$ m_1 N + m_2 a_k = t_1 = gcd(N, a_k) $. Nyní můžeme vytvořit matici

\begin{align*}
U =
    \begin{pmatrix}
        m_1         & m_2      \\
        -sa_k / t_1 & sN / t_1 \\
    \end{pmatrix}
,
\end{align*}
kde $ s \in \{1, -1\} $ je zvoleno tak, aby
$ t_2 = (-sa_k / t_1) \bar{N} + (sN / t_1) \bar{a}_k $ bylo kladné. Matice $ U $
je unimodulární, neboť
\begin{align*}
\det U =
    \begin{vmatrix}
        m_1         & m_2      \\
        -sa_k / t_1 & sN / t_1 \\
    \end{vmatrix}
    = \frac{s(m_1 N + m_2 a_k)}{t_1} = \frac{s t_1}{t_1} = \pm 1
.
\end{align*}
A konečně můžeme zkonstruovat matici

\begin{align*}
Q =
    \begin{pmatrix}
        1 &        &   &   &   &        &   \\
          & \ddots &   &   &   &        &   \\
          &        & 1 &   &   &        &   \\
          &        &   & U &   &        &   \\
          &        &   &   & 1 &        &   \\
          &        &   &   &   & \ddots &   \\
          &        &   &   &   &        & 1 \\
    \end{pmatrix}
,
\end{align*}
která, jak plyne z předchozího, bude také unimodulární. Aplikací $ Q $ na matici
$ CB $ dostáváme
\begin{align*}
QCB =
    \begin{pmatrix}
        \ast   & \ast      \\
        \vdots & \vdots    \\
        \ast   & \ast      \\
        t_1    & \ast      \\
               & t_2       \\
        b_1    & \bar{b}_1 \\
        \vdots & \vdots    \\
        b_k    & \bar{b}_k \\
    \end{pmatrix}
\end{align*}
a platí $ t_1 \mid b_i$ kde $ i = 1, \dots, k $. Můžeme tedy snadno vyeliminovat
prvky pod $ t_1 $. Následně snadno provedeme redukci prvků nad $ t_1 $ a
konečně i prvků nad a pod $ t_2 $. To lze provést pomocí elementárních řádkových
operací, které můžeme odpovídajícím způsobem zapsat do matice $ Q $. Takto
upravená matice $ Q $ již bude splňovat podmínky věty a důkaz je tak hotov.

\end{proof}




\section{RST algoritmus}
V následujícím textu využijeme výše popsanou proceduru Sloupcové redukce k vytvoření
RST algoritmu, který na vstupu bere $ n' \times m' $ vstupní matici $ A' $ a
vrací její redukovaný schodovitý tvar včetně transformační matice. Nejdříve však
musíme definovat pojem rank profile (nejsme si vědomi existence vhodného českého
ekvivalentu, proto budeme používat původní anglický výraz).

\begin{defi}
Buď $ A $ matice $ n \times m $ a nechť $ r $ značí hodnost matice $ A $. Nechť
$ G $ je reprezentace matice $ A $ ve schodovitém tvaru. Pod pojmem
\textbf{rank profile} pak rozumíme uspořádanou $ r $-tici $ (j_1, \dots, j_r) $, kde
$ j_i $ je sloupcový index prvního nenulového prvku v $ i $-tém řádku matice $ G $.
\end{defi}

Abychom se vyhnuli ošetřování množství speciálních případů (například matice
mající hodnost $ 0 $ a podobně), budeme namísto matice $ A' $ uvažovat matici

\begin{align} \label{input_A}
A =
    \begin{pmatrix}
        1 &    &   \\
          & A' &   \\
          &    & 1 \\
    \end{pmatrix}
.
\end{align}
Poznamenejme, že takováto matice bude mít rank profile ve tvaru
$ (1, j_2, \dots, j_{r - 1}, m) $, kde $ r \geq 2 $ je hodnost matice $ A $ a
$ n \times m $ jsou její rozměry.

Nyní budeme definovat RST algoritmus. Pro názornost nejdříve uvedeme variantu,
která potřebuje dopředu znát rank profile. Ten je možné spočítat například
Gausovou eliminační metodou. Poznamenejme, že Gausova eliminace spadá do složitostní
třídy $ \mathcal{O}(n^3) $, což je zanedbatelné vzhledem k celkové složitosti
našeho algoritmu. Přesto však později uvedeme také jednoduchou modifikaci RST
algoritmu, která již rank profile nevyžaduje. Nyní přistupme k samotné definici.

\begin{algorithm}
{
    \textbf{Algoritmus:} Výpočet redukovaného schodovitého tvaru
}

\KwData{
    Celočíselná $ n \times m $ matice $ A $ mající rank profile
    $ (j_1, \dots, j_r) $, kterou lze zapsat jako v \ref{input_A}.
}
\KwResult{
    Matice $ Q, C $ a $ T $ splňující $ QCA = T $, kde $ Q $ a $ C $ jsou
    unimodulární a $ T $ má prvních $ m - 1 $ sloupců v redukovaném schodovitém
    tvaru.
}
\Begin{
    $ Q \leftarrow I_n $\;
    $ C \leftarrow I_n $\;
    $ T \leftarrow A $\;
    % \vspace{\baselineskip}
    \For{ $k \leftarrow 2$ \KwTo $ r - 1 $} {
        $ B_k \leftarrow n \times 2 $ matice $ (col(T, j_k) \mid col(T, j_{k + 1})) $\;
        $ (Q_k, C_k) \leftarrow $ \textbf{ColumnReduction}$ (B_k, n - k - 1) $\;
        $ Q \leftarrow Q_k C_k Q C_{k}^{-1} $\;
        $ C \leftarrow C_k C $\;
        $ T \leftarrow Q_k C_k T $\;
    }
    \Return{$ (Q, C, T) $}
}

\end{algorithm}

\begin{vet} \label{RST_algo}
Algoritmus pro výpočet redukovaného schodovitého tvaru je korektní.
\end{vet}
\begin{proof}
\end{proof}


\cleardoublepage

